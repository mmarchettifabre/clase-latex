%%
%% Este es un archivo .tex que debe ser compilado con pdflatex
%% Para mas informacion a cerca de Latex:
%% https://www.latex-project.org/
%% Tutorial en español:
%% http://metodos.fam.cie.uva.es/~latex/



\documentclass[12pt]{beamer}
\usepackage[spanish]{babel}
\usepackage{hyperref}
\usepackage{soul}

\mode<presentation>{
  \usetheme{Madrid} %default,boxes,Bergen,Boadilla,Madrid,AnnArbor,CambridgeUS,Pittsburgh,Rochester,Antibes,JuanLesPins,Berkeley,PaloAlto,Goettingen,Marburg,Hannover,Berlin,Ilmenau,Dresden,Darmstadt,Frankfurt,Singapore,Szeged,Copenhagen,Luebeck,Malmoe,Warsaw
%%\setbeamercovered{transparent}
}
\setbeamercolor{postit}{bg=yellow!50!white}

\title{\textrm{\LaTeX}}
\author{Martin Marchetti Fabre}

\begin{document}

\frame{\titlepage}


\begin{frame}%[Opciones]
%%  \begin{block}{Contenido}
  \frametitle{Contenido}
  \begin{itemize}
    \item<2-> ¿Qué es \textrm{\LaTeX}?
    \item<3-> ¿Por qué debería usarlo?
    \item<4-> ¿Qué necesito para usarlo?
    \item<5-> \only<5>{¿Esto con que se come?}\only<6->{¿\st{Esto con que se come?} ¿Cómo lo uso?}
    \item<7> Manos al código
    \end{itemize}
%%  \end{block}
\end{frame}

\begin{frame}[c]
  \frametitle{¿Qué es \textrm{\LaTeX}?}

  \begin{block}{}
    Es un sistema de composición de textos, orientado a la creación de documentos escritos que presenten una alta calidad tipográfica.
  \end{block}
  \onslide<3->{  \hspace{4cm}
    \begin{beamercolorbox}[wd={4cm},sep=1mm,center,rounded=true,shadow=true]{postit}
      \st{``lo que ves es lo que obtienes''}
    \end{beamercolorbox}
  }

  \onslide<4->{  \hspace{4cm}
    \begin{beamercolorbox}[wd={4cm},sep=1mm,center,rounded=true,shadow=true]{postit}
      ``lo que ves es lo que quieres decir''
    \end{beamercolorbox}
  }

\end{frame}


\begin{frame}%[Opciones]
  \frametitle{¿Por qué debería usarlo?}
  \pause
  \begin{block}{}
    Los documentos generados con \textrm{\LaTeX{}} tienen calidad \textit{profesional}. Esto se puede observar en documentos que contengan formulas y ecuaciones. Pero cualquier tipo de documento puede aprovechar la calidad de imprenta de \textrm{\LaTeX{}}
  \end{block}\pause
  \begin{block}{}
    \textrm{\LaTeX{}} permite separar el contenido y el formato del documento. El escritor se concentra en el \textit{``qué''} y \textrm{\LaTeX{}} se encarga del \textit{``cómo''}. Muchas tareas son generadas de forma automatica: \textbf{numerar capítulos y figuras, incluir y organizar la bibliografía adecuada, mantener índices y referencias cruzadas}
  \end{block}
\end{frame}

\begin{frame}%[Opciones]
  \frametitle{¿Qué necesito para usarlo?}
  \begin{block}{Las 3 partes de \textrm{\LaTeX{}}}
    \begin{itemize}
    \item<2-> Motor de \textrm{\LaTeX{}}
    \item<3-> Editor de textos
    \item<4-> Visor de documentos
    \end{itemize}
  \end{block}
\end{frame}

\begin{frame}%[Opciones]
  \frametitle{¿Qué necesito para usarlo en Windows?}
  \begin{block}{Motor de \textrm{\LaTeX{}}}
    \begin{itemize}
    \item MiKTeX
    \end{itemize}
  \end{block}
  \begin{block}{Editor de textos}
    \begin{itemize}
    \item Sublime Text
    \item TexStudio
    \item Lyx
    \item TeXnicCenter
    \end{itemize}
  \end{block}
  \begin{block}{Visor de documentos}
    \begin{itemize}
    \item Adobe Reader
    \end{itemize}
  \end{block}
\end{frame}

\begin{frame}%[Opciones]
  \frametitle{¿Qué necesito para usarlo en Linux?}
  \begin{block}{Motor de \textrm{\LaTeX{}}}
    \begin{itemize}
    \item TexLive
    \end{itemize}
  \end{block}
  \begin{block}{Editor de textos}
    \begin{itemize}
    \item Sublime Text
    \item TexStudio
    \item Lyx
    \item Emacs
    \end{itemize}
  \end{block}
  \begin{block}{Visor de documentos}
    \begin{itemize}
    \item Evince
    \item Okular
    \end{itemize}
  \end{block}
\end{frame}

\begin{frame}%[Opciones]
  \frametitle{¿Qué necesito para usarlo en Mac OS X?}
  \begin{block}{MacTex}
    MacTex es una distribución que incluye todos los componentes necesarios para trabajar con \textrm{\LaTeX{}}
  \end{block}
  \begin{block}{Motor de \textrm{\LaTeX{}}}
    \begin{itemize}
    \item MacTex
    \end{itemize}
  \end{block}
  \begin{block}{Editor de textos}
    \begin{itemize}
    \item TexShop
    \item Lyx
    \end{itemize}
  \end{block}
  \begin{block}{Visor de documentos}
    \begin{itemize}
    \item TexShop
    \item Preview
    \end{itemize}
  \end{block}
\end{frame}

\begin{frame}%[Opciones]
  \frametitle{¿Cómo lo uso?}\pause
  \begin{block}{Editar}
    El primer paso consiste en crear un archivo \texttt{.tex} que contiene el código describiendo la estructura y el contenido del documento.
  \end{block}\pause
  \begin{block}{Compilar}
    Despues, hay que convertir el archivo \texttt{.tex} en un documento con formato \texttt{.pdf} que se puede imprimir y ver en pantalla.
  \end{block}\pause
  \begin{block}{Visualizar}
    Una vez compilado el documento, y si no hubieron \textbf{errores}, puedes visualizarlo para ver si todo está correcto.
  \end{block}
\end{frame}

\begin{frame}%[Opciones]
  \frametitle{Manos al código}
  \framesubtitle{Estructura general}\pause
  \begin{block}{Preambulo}
    Declaraciones de carácter GLOBAL que afectan a la totalidad del documento.
    \begin{description}
    \item[\texttt{\textbackslash documentclass[opciones]\{tipo\_de\_documento\}}]

      Esta linea es \textbf{OBLIGATORIA}, ya que define el tipo de documento o clase que vamos a escribir.

    \item[\texttt{\textbackslash usepackage[opciones]\{paquete\}}]

      Carga paquetes de utilidades (para incluir graficos, tesxto en colores, hipervinculos, etc...)
    \end{description}
  \end{block}\pause
  \begin{block}{Cuerpo}
    Todo lo comprendido entre \texttt{\textbackslash begin\{document\}} y \texttt{\textbackslash end\{document\}}, que es el documento propiamente dicho.
  \end{block}\pause
\end{frame}

\begin{frame}%[Opciones]
  \frametitle{Manos al código}
  \framesubtitle{Clase \texttt{article}}
  \begin{block}{}
    \texttt{\textbackslash documentclass[a4,11pt]\{article\}\\
      \textbackslash usepackage[spanish]\{babel\}\\
      \bigskip
      \textbackslash title\{Hola Mundo!\}\\
      \textbackslash author\{Martin\}\\
      \bigskip
      \textbackslash begin\{document\}\\
      \textbackslash maketitle\\
      \bigskip
      Mi primer texto en latex.\\
      \bigskip
      \textbackslash end\{document\}
    }
\end{block}
\end{frame}

\begin{frame}%[Opciones]
  \frametitle{Manos al código}
  \framesubtitle{Clase \texttt{beamer}}
  \begin{block}{}
    \texttt{\textbackslash documentclass[12pt]\{beamer\}\\
      \textbackslash usepackage[spanish]\{babel\}\\
      \bigskip
      \textbackslash title\{Hola Mundo!\}\\
      \textbackslash author\{Martin\}\\
      \bigskip
      \textbackslash begin\{document\}\\
      \textbackslash frame\{\textbackslash titlepage\}\\
      \bigskip
      \textbackslash begin\{frame\}\\
      Mi primera presentación con Beamer.\\
      \textbackslash end\{frame\}\\
      \bigskip
      \textbackslash end\{document\}
    }
\end{block}
\end{frame}

\begin{frame}%[Opciones]
  \frametitle{Manos al código}
  \framesubtitle{Repositorio en GitHub}
  \begin{block}{Todo el código lo podemos encontrar en...}
    \url{http://www.latex-project.org/}
  \end{block}
\end{frame}

\end{document}
