\documentclass[a4,11pt]{article}
%\usepackage[latin1]{inputenc}
%esto es un comentario y no se imprime ni se compila
\usepackage[T1]{fontenc}
\usepackage[spanish]{babel}
%\renewcommand{\shorthandsspanish}{}

\usepackage{graphicx}

\title{Documento Fuente \LaTeX{}}
\author{Yo}


\begin{document}
\maketitle
%Solo con la siguiente linea creamos todo el índice
\tableofcontents

\begin{abstract}
  Ejmplo de documento \LaTeX{} de la clase {\ttfamily article} con una estructura reducida. Ésta incluye secciones, subsecciones y una referencia cruzada.
\end{abstract}

\section{Primera sección} \label{primera}
Una primera sección con una fórmula y una lista.

\subsection{Fórmula}
Una fórmula usando el entorno \texttt{equation}:

\begin{equation}
  \sum_{n=1}^{\infty}       \frac{1}{n^{2}} = \frac{\pi^{2} }{6}
  \label{sumatoria}
\end{equation}

Una ecuación en la misma linea usando el simbolo \texttt{\$}: $\frac{\partial f}{\partial x} =\frac{\partial f}{\partial y}$, a diferencia de la equación \ref{sumatoria}, ésta no se puede referenciar.
Para generar ecuaciones se puede usar la web: https://latex.codecogs.com/legacy/eqneditor/editor.php
\subsection{Listas}
Una lista de ítems señalados con una marca:

\begin{itemize}
\item Primer ítem
\item Segundo ítem
\item Tercer ítem
\end{itemize}

\section{Segunda sección}

Ésta sección complementa a la sección \ref{primera} incluyendo ejemplos de tablas (Ver tabla \ref{tab:importaciones}) y figuras (Ver figura \ref{fig:gnutux}) escritas en \LaTeX.

\subsection{Tabla}
\begin{table}[h]
  \begin{center}
    \begin{tabular}{ccc}
      Pais   &Carne&Verduras\\ \hline
      España &1390 &980     \\
      Francia&1504 &3020    \\
      Italia &2010 &1040    \\
    \end{tabular}
  \end{center}
  \caption{Importaciones (en millones de €) de carne y verduras}\label{tab:importaciones}
\end{table}


\subsection{Figuras}
\begin{figure}[h]
  \begin{center}
    \includegraphics[width=10cm]{Logo-ubuntu_no(r)-black_orange-hex.pdf}
  \end{center}
  \caption{Logo de Ubuntu}\label{fig:ubuntu}
\end{figure} 


\subsection{Figuras}
\begin{figure}[h]
  \begin{center}
    \includegraphics[width=10cm]{Gnulinux.pdf}
  \end{center}
  \caption{Figura de Tux y GNU}\label{fig:gnutux}
\end{figure} 

\end{document}