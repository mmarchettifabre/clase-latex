%%
%% Este es un archivo .tex que debe ser compilado con pdflatex
%% Para mas informacion a cerca de Latex:
%% https://www.latex-project.org/
%% Tutorial en español:
%% http://metodos.fam.cie.uva.es/~latex/



\documentclass[12pt]{beamer}
\usepackage[spanish]{babel}
\usepackage{hyperref}

\mode<presentation>{
  \usetheme{Madrid} %default,boxes,Bergen,Boadilla,Madrid,AnnArbor,CambridgeUS,Pittsburgh,Rochester,Antibes,JuanLesPins,Berkeley,PaloAlto,Goettingen,Marburg,Hannover,Berlin,Ilmenau,Dresden,Darmstadt,Frankfurt,Singapore,Szeged,Copenhagen,Luebeck,Malmoe,Warsaw
  \setbeamercovered{transparent}
}

\title{Las herramientas del Software libre}
\author{Martin Marchetti Fabre}

\begin{document}

\frame{\titlepage}


\begin{frame}%[Opciones]
%%  \begin{block}{Contenido}
  \frametitle{Contenido}
  \begin{itemize}
    \item<2-> Sobre mí.
      \begin{itemize}
      \item<3-> Trayectoria laboral.
      \item<4-> Desempeño laboral actual.
      \end{itemize}
    \item<5-> Software libre.    
      \begin{itemize}
      \item<6-> Las 4 libertades.             
      \item<7-> Licencias y \textit{copyleft}.
      \item<8-> Conclusiones.
      \item<9-> Herramientas.
      \end{itemize}
    \end{itemize}
%%  \end{block}
\end{frame}

\begin{frame}%[Opciones]
  \frametitle{Sobre mí}
  \begin{itemize}
  \item Nací en Villa del Rosario. 
  \item Soy Ingeniero Electrónico, egresado de la UTN-FRC.
  \item Trabajo como Ingeniero Senior en INPHI, en el diseño de hardware de sistemas de procesamiento para comunicaciones digitales.
  \end{itemize}
\end{frame}


\begin{frame}%[Opciones]
  \frametitle{Sobre mi trabajo}
  \begin{center}
    \includegraphics[height=3.5cm]{inphi.jpg}
    \includegraphics[height=3.5cm]{clariphy.jpeg}
  \end{center}
\end{frame}

\begin{frame}%[Opciones]
  \frametitle{Sobre mi trabajo}
  \begin{itemize}
  \item Diseño, modelado, implementación y verificación de algoritmos de DSP para transceptores ópticos coherentes de bajo consumo.
  \item Reducción de consumo de potencia del diseño utilizando optimizaciones algorítmicas y técnicas de implementación digital.
  \item Diseño y ejecución de verificación a nivel de algoritmo, RTL y celdas de implementación.
  \item Experiencia en diseño HDL con aritmetica de punto fijo.
  \end{itemize}
\end{frame}

\begin{frame}%[Opciones]
%%  \begin{block}{Contenido}
  \frametitle{Contenido}
  \begin{itemize}
    \item<0> Sobre mí.
      \begin{itemize}
      \item<0> Trayectoria laboral.
      \item<0> Desempeño laboral actual.
      \end{itemize}
    \item<1> Software libre.    
      \begin{itemize}
      \item<1> Las 4 libertades.
      \item<1> Licencias y \textit{copyleft}.
      \item<1> Conclusiones.
      \item<1> Herramientas.
      \end{itemize}
    \end{itemize}
%%  \end{block}
\end{frame}

\begin{frame}
  \frametitle{Software libre}
  \begin{itemize}
  \item \texttt{«Software libre»} es el software que respeta la libertad de los usuarios y la
    comunidad.
  \item Los usuarios tienen la libertad de {\bfseries ejecutar, copiar, distribuir, estudiar,
    modificar y mejorar el software.}
  \item El \texttt{«software libre»} es una cuestión de libertad, no de precio.
  \item Con estas libertades, los usuarios controlan el programa y lo que este hace.
  \item Cuando los usuarios no controlan el programa, decimos que dicho programa \texttt{«no es
    libre»}, o que es \texttt{«privativo»}.
  \item Un programa que no es libre controla a los usuarios, y el programador controla el programa.
  \end{itemize}  
\end{frame}

\begin{frame}
  \frametitle{Software libre}
  \framesubtitle{Las 4 libertades escenciales}
  \onslide<1,6>{\begin{block}{\small Libertad 0}
    \footnotesize La libertad de ejecutar el programa como se desee, con cualquier propósito.
  \end{block}}
  \onslide<2,5->{\begin{block}{\small Libertad 1}
      \footnotesize La libertad de estudiar cómo funciona el programa, y cambiarlo para que haga lo que usted quiera.
      \only<5>{\color{red}Se necesita acceso al código fuente.}
      \only<6>{\color{black}Se necesita acceso al código fuente.}
  \end{block}}
  \onslide<3,6>{\begin{block}{\small Libertad 2}
    \footnotesize La libertad de redistribuir copias para ayudar a otros.
  \end{block}}
  \onslide<4->{\begin{block}{\small Libertad 3}
    \footnotesize La libertad de distribuir copias de sus versiones modificadas a terceros.
    Esto le permite ofrecer a toda la comunidad la oportunidad de beneficiarse de las modificaciones.
    \only<5>{\color{red}Se necesita acceso al código fuente.}
    \only<6>{\color{black}Se necesita acceso al código fuente.}
  \end{block}}
\end{frame}

\begin{frame}
  \frametitle{Licencias y \textit{copyleft}}
  \begin{itemize}
  \item Se pueden establecer reglas sobre la manera de compartir y distribuir software libre, siempre
    que no entren en conflicto con las libertades principales.
  \item La manera más simple de hacer que un programa sea software libre consiste en ponerlo
    en el dominio público, sin copyright.
  \item El \textit{copyleft} es un método general para liberar un programa u otro tipo de trabajo,
    que requiere que todas las versiones modificadas y extendidas sean también libres.
  \item Existen diversos tipos de licencias \textit{copyleft}, como Licencia Pública General de GNU,
    entre muchas otras.
    
  \end{itemize}  
\end{frame}

\begin{frame}
  \frametitle{Licencias y \textit{copyleft}}
  \begin{center}
    \only<1>{\includegraphics[height=7cm]{category.es.pdf}}
    \only<2>{\includegraphics[height=7cm]{free.high.category.es.pdf}}
    \only<3>{\includegraphics[height=7cm]{priv.high.category.es.pdf}}
    \only<4>{\includegraphics[height=7cm]{open.high.category.es.pdf}}
    \only<5>{\includegraphics[height=7cm]{public.high.category.es.pdf}}
    \only<6>{\includegraphics[height=7cm]{laxo.high.category.es.pdf}}
    \only<7>{\includegraphics[height=7cm]{copyleft.high.category.es.pdf}}
    \only<8>{\includegraphics[height=7cm]{gpl.high.category.es.pdf}}
    \only<10>{\includegraphics[height=7cm]{share.high.category.es.pdf}}
    \only<9>{\includegraphics[height=7cm]{gratis.high.category.es.pdf}}
  \end{center}

\end{frame}

\begin{frame}
  \frametitle{Conclusiones}
  \onslide<1->{\begin{block}{\small SaaSS}
      \scriptsize Debemos evitar los «Servicios sustitutivos del software», que significa que un
      servidor ajeno realiza las tareas informáticas del usuario.
  \end{block}}
  \onslide<2->{\begin{block}{\small Injusticias primarias y Secundarias}
      \scriptsize Cuando usamos software privativo nos perjudicamos a nosotros mismos, pero ademas
      perjudicamos a los demas de forma indirecta, ya sea no compartiendo software o forzando a
      los demas a usar determinado software. 
  \end{block}}
  \onslide<3->{\begin{block}{\small Estado}
      \scriptsize Los entes públicos existen para los ciudadanos, no para sí mismos.
      Tienen el deber de conservar el control absoluto sobre las tareas informaticas a fin de
      garantizar su correcta ejecución en beneficio de los ciudadanos. 
  \end{block}}
  \onslide<4>{\begin{block}{\small Educación}
      \scriptsize Las escuelas influyen sobre el futuro de la sociedad a través de lo que enseñan.
      Para que esta influencia sea positiva, deben enseñar exclusivamente software libre.
      Enseñar el uso de un programa privativo equivale a imponer la dependencia,
      que es lo contrario de la misión educativa.
  \end{block}}

\end{frame}

\begin{frame}
  \frametitle{Conclusiones}
  \begin{block}{}
    Todos merecemos tener el control de nuestra propia actividad informática.
  \end{block}
  \begin{block}{¿Cómo podemos conseguirlo?}
    \begin{itemize}
    \item Rechazando el software que no es libre en los ordenadores que nos pertenecen o que usamos
      regularmente, y rechazando el SaaSS.
    \item Desarrollando software libre
    \item Rehusando desarrollar o promover software privativo o el SaaSS.
    \item Difundiendo estas ideas.
    \end{itemize}
  \end{block}

\end{frame}

\begin{frame}
  \frametitle{Herramientas}
  \begin{columns}
    \begin{column}{5cm}
      %\vspace*{1cm}
      \only<1-7>{\begin{block}{SO GNU/Linux}
        \small
        \begin{itemize}
        \item<2- |alert@2> Ubuntu
        \item<3- |alert@3> Debian
        \item<4- |alert@4> Red Hat
        \item<5- |alert@5> Mint
        \item<6- |alert@6> Fedora
        \item<7- |alert@7> Arch
        \end{itemize}
      \end{block}}

      \only<8-11>{\begin{block}{SO GNU/Linux totalmente libres}
        \small
        \begin{itemize}
        \item<9- |alert@9> PureOS
        \item<10- |alert@10> Trisquel
        \item<11- |alert@11> Hyperbola
        \end{itemize}
      \end{block}}
      
      \only<12-19>{\begin{block}{Software para el hogar}
        \small
        \begin{itemize}
        \item<12- |alert@12> Chromiun
        \item<13- |alert@13> Iceweasel-Mozilla Firefox
        \item<14- |alert@14> LibreOffice
        \item<15- |alert@15> Emacs
        \item<16- |alert@16> \LaTeX
        \item<17- |alert@17> Gimp
        \item<18- |alert@18> Inkscape
        \item<19- |alert@19> VLC
        \end{itemize}
      \end{block}}

      \only<20-24>{\begin{block}{Herramientas de programacion}
        \small
        \begin{itemize}
        \item<20- |alert@20> GCC
        \item<21- |alert@21> Eclipse
        \item<22- |alert@22> Anjuta
        \item<23- |alert@23> NetBeans IDE
        \item<24- |alert@24> Python
        \end{itemize}
      \end{block}}

      \only<25-27>{\begin{block}{Herramientas para servidores}
        \small
        \begin{itemize}
        \item<25- |alert@25> Apache HTTP Server
        \item<26- |alert@26> MySQL-MariaDB
        \item<27- |alert@27> PHP
        \end{itemize}
      \end{block}}

    \end{column}
    \begin{column}{5cm}
      \begin{center}
        \only<1>{\includegraphics[width=4cm]{Gnulinux.pdf}}
        \only<2>{\includegraphics[width=4cm]{Logo-ubuntu_no(r)-black_orange-hex.pdf}}
        \only<3>{\includegraphics[width=3.5cm]{Debian-OpenLogo.pdf}}
        \only<4>{\includegraphics[width=5.5cm]{red-hat-logo1.jpg}}
        \only<5>{\includegraphics[width=3.5cm]{Logo_Linux_Mint.png}}
        \only<6>{\includegraphics[width=5cm]{Fedora_logo_and_wordmark.pdf}}
        \only<7>{\includegraphics[width=5cm]{archlinux-logo-dark-90dpi.ebdee92a15b3.png}}
        \only<9>{\includegraphics[width=5cm]{pureos.jpg}}
        \only<10>{\includegraphics[width=5cm]{Logo-Trisquel.pdf}}
        \only<11>{\includegraphics[width=5cm]{Hyperbola_GNU+Linux-libre_logo.pdf}}
        \only<12>{\hspace*{0.7cm}\includegraphics[width=3.5cm]{Chromium_11_Logo.pdf}}
        \only<13>{\includegraphics[width=3.5cm]{Iceweasel_icon.pdf}}
        \only<13>{\hspace*{0.4cm}\includegraphics[width=3.5cm]{Firefox_logo,_2019.pdf}}
        \only<14>{\includegraphics[width=5cm]{LibreOffice_Logo_Flat.pdf}}
        \only<15>{\hspace*{0.7cm}\includegraphics[width=4cm]{EmacsIcon.pdf}}
        \only<16>{\includegraphics[width=5cm]{LaTeX_logo.pdf}}
        \only<17>{\includegraphics[width=5cm]{The_GIMP_icon_-_gnome.pdf}}
        \only<18>{\includegraphics[width=5cm]{Inkscape_Logo.pdf}}
        \only<19>{\includegraphics[width=5cm]{VLC_Icon.pdf}}
        \only<20>{\includegraphics[width=5cm]{GNU_Compiler_Collection_logo.pdf}}
        \only<21>{\includegraphics[width=5cm]{Eclipse-Luna-Logo.pdf}}
        \only<22>{\includegraphics[width=5cm]{Anjuta.pdf}}
        \only<23>{\includegraphics[width=5cm]{Apache_NetBeans_Logo.pdf}}
        \only<24>{\includegraphics[width=5cm]{Python-logo-notext.pdf}}
        \only<25>{\includegraphics[width=5cm]{Apache_HTTP_server_logo_(2016).png}}
        \only<26>{\includegraphics[width=3.5cm]{mysqloptimizar1.png}}
        \only<26>{\includegraphics[width=3.5cm]{MariaDB_Logo.png}}
        \only<27>{\includegraphics[width=5cm]{PHP-logo.pdf}}
      \end{center}
    \end{column}
  \end{columns}
  
\end{frame}

\begin{frame}
  \frametitle{Fuentes}
  \url{https://www.gnu.org/philosophy/free-sw.es.html}
  \url{https://www.gnu.org/philosophy/free-software-even-more-important.html}
  \url{https://es.wikipedia.org/wiki/GNU/Linux#Distribuciones}
  
\end{frame}

\end{document}
